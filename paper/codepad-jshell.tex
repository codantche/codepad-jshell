\documentclass{article}
\usepackage{graphicx} 
\usepackage[T1]{fontenc}
\usepackage{lmodern}
\usepackage{plantuml}
\usepackage{url}

\usepackage{listings}
\usepackage{xcolor}

\lstdefinestyle{jshell}{
  language=Java,
  basicstyle=\ttfamily\small,
  backgroundcolor=\color{gray!5},
  frame=single,
  framerule=0.5pt,
  rulecolor=\color{gray!30},
  numbers=left,
  numberstyle=\tiny\color{gray},
  xleftmargin=1em,
  showstringspaces=false,
  tabsize=2,
  columns=flexible,
  moredelim=**[is][\color{blue}]{@}{@}, % for blue prompt like @jshell>
}

\lstset{style=jshell}

\title{Comparing Code Pad and JShell Features: A Pragmatic Pedagogical Perspective}
\author{Marco Mangan\thanks{\texttt{marco.mangan@gmail.com}}}
\date{October 2025}

\begin{document}

\maketitle

\begin{abstract}
Learning introductory programming remains a challenge, particularly when students are required to use professional tools. BlueJ, designed under the objects-first approach, provides a simplified environment and an interactive feature known as the Code Pad, which supports exploratory learning. Since the introduction of JShell in JDK 9, Java developers have gained a professional read–eval–print–loop (REPL) tool that overlaps with some of BlueJ’s interactive capabilities. This paper investigates whether JShell can offer a learning experience comparable to BlueJ’s Code Pad in the context of introductory computer science education. Through feature-oriented modeling, benchmark examples, and a review of related literature, we identify key similarities and differences between the two tools. The results suggest that while JShell enables more powerful and flexible interactions, BlueJ retains pedagogical advantages due to its visual modeling support and simplified interface. The study contributes to understanding how professional and educational environments can converge to support diverse student needs in programming education.
\end{abstract}

\section{Introduction}

This work presents a pragmatic, pedagogical comparison between BlueJ’s \textit{Code Pad} and Java’s \textit{JShell}. Rather than focusing on instructional theory, the study examines how students engage with the features that are effectively available in each environment while performing introductory programming tasks.

Learning to program computers is hard. By the late 1990s, the growing need to teach object-oriented programming sparked a broad discussion on instructional approaches. Historically, programmers developed their skills by progressing from computer architecture to assembly and then to early programming languages. Within this evolution, the understanding of objects typically came later.

With the growing prevalence of object-oriented languages such as C++ and Java, an \textit{objects-first} approach to teaching programming was advocated at that time. Before this shift, languages like BASIC, Smalltalk, and LISP were often learned directly—without prior exposure to other languages or to hardware concepts. Programming in these contexts occurred at a higher level of abstraction, and the \textit{objects-first} approach emerged as a natural extension of this perspective.

\textit{BlueJ} is a programming environment based on the \textit{objects-first} learning approach, first released in 1999 \cite{barnes2008objects, bluej_paper}. Among its many features, the \textit{Code Pad} was an innovation that allowed Java programmers to experiment with and explore commands without the need to write a complete application.

\textit{JShell} is a read–eval–print loop (REPL) introduced in JDK~9, around 2017 \cite{jshell_docs}. Some of its features are clearly similar to those found in BlueJ’s \textit{Code Pad} module and in earlier educational programming environments.

This resemblance raises the question of whether the interactive features once unique to BlueJ’s \textit{Code Pad} are still necessary in introductory programming, now that \textit{JShell} provides similar capabilities within the standard JDK.

Using \textit{JShell} may be a matter of personal preference, but to answer this question objectively we need to compare the features of \textit{Code Pad} and \textit{JShell}. This study adopts a pragmatic pedagogical perspective to identify and analyze the interactive features available in both environments.

The paper draws on feature-oriented modeling, benchmark examples, and literature review to identify key similarities and differences. The results are intended to inform educators and tool designers about how professional and educational environments can converge to better support introductory programming.

Ultimately, we need to better define what “education” means in programming, as professional developers never cease to learn. 

\section{Motivation}

Regardless of pedagogical foundations or teaching philosophy, students respond primarily to the tools actually at hand. Their exploratory behavior is therefore shaped more by the affordances of the environment—what the interface allows them to do—than by abstract learning principles. In this sense, tools such as BlueJ’s \textit{Code Pad} and Java’s \textit{JShell} not only mediate how students write code but also how they conceptualize programming itself.

This study is motivated by classroom observations in an undergraduate Computer Science course following the adoption of Java, BlueJ, and the \textit{objects-first} approach. Over several semesters, students have displayed distinct patterns of exploration and inquiry depending on the environment available to them.

Since the adoption of that approach, more than two decades have passed. During this period, three main factors have been observed: (a) a significant increase in computer literacy, (b) broader access to new devices and software, and (c) a shift in the very notion of what a computer program is.

\textit{BlueJ} is a desktop application whose look and feel contrast with that of contemporary software. In particular, its multiple-document interface has become one of the most distracting factors during classes.

Students often prefer other programming environments, most notably Microsoft Visual Studio Code. In some cases, there is even peer pressure from senior students to move directly to a more professional editor.

%Other editors include simple text editors, like Geany. No matter the editor, we can be sure that the Java Development Kit, which includes the Java Shell, is available.

Visual Studio Code can be run locally or in the cloud through a web browser. This flexibility allows students to use tablet devices while providing a more familiar and professional-looking interface. Some students bring their own laptops or tablets, but most have access to more than one computer. The cloud-based version of VS Code eliminates the need for separate Java and BlueJ installations and keeps all projects organized within the same cloud folder.

In this context, there is a clear need to provide a BlueJ-like experience for students who choose not to use the BlueJ environment.

\section{Research Question}

The central research question of this study is broad: can a BlueJ-like learning experience be achieved using professional tools such as Java’s JShell?

BlueJ includes many pedagogically grounded features. Therefore, it is necessary to compare the learning experience associated with each feature. This study focuses on the interactive and exploratory learning experience offered by BlueJ’s Code Pad.

Therefore, the main research question of this study is:
Can the use of Java’s JShell provide students with a learning experience comparable to the one offered by BlueJ’s Code Pad?


Even with this focus, we need to use sub-features to allow a direct comparison. By doing so, we provide a list of pragmatic features  that would help to understand the learning experience offered by both programs.

%, one of the many BlueJ pedagogically found features

%JShell was not available at the time that the BlueJ was proposed. Read-eval-loop shells were available since the adoption of interpreted languages, including the Unix bash shell, BASIC, and popular languages like Python and ECMAScript.

To answer this question, the study combines feature-oriented modeling, benchmark examples, and a review of related literature. This methodological combination allows a pragmatic and pedagogically informed comparison between BlueJ’s Code Pad and Java’s JShell, focusing on the interactive features that shape students’ learning experiences.


\section{Methodology}

Literature review

Benchmark with examples

Feature Oriented Domain Models

Comparing both programs


Most evidences are available on the paper repository \cite{mangan2025codepadjshell}.

\section{Student Expectations}

Get a job fast

Use professional tools

Write professional programs: mobile and Web clients, Web applications, games.

\section{Usage Scenarios}

Exercises from Barnes and Kolling

\subsection{Evaluate expressions}

Following the examples of the BlueJ book, the first example includes the evaluation of an arithmetic expression and a method call.

BlueJ return value and type of each expression. 
The internal structure of the objects can be visually inspected in BlueJ. 

JShell returns only the value. Data types can be consulted using the command /vars.
JShell users must recur to introspection commands, which would be out of the scope of a first programming course. 
The CLI of Java Shell at least allow the use of ids to refer to expression values,  objects  and basic types.

\begin{lstlisting}
jshell> 4 + 5
$1 ==> 9
\end{lstlisting}

Method call
\begin{lstlisting}
jshell> "Wombat".substring(3, 5)
$2 ==> "ba"
\end{lstlisting}
Two-way integration with visual object model

After each compile

Importing packages

Variable declaration

Saving scripts

Conditional commands


\section{Pragmatic Feature Comparison}
Commonality and variability are

The comparison presented here focuses on what learners can directly experience in practice, rather than on the theoretical intent of each design.

\section{Related Work}

At least on study points out the inadequacy of the Java Shell in the classroom
\cite{politz_minnes2018jshell}.

\section{Discussion}

JShell is a professional tool

BlueJ integration of many features.

BlueJ features not provided by Java Shell include class and object visual models and their integration with Code Pad.

In order to compare the potential experience, we compare features of the applications.

The need for a in-classroom is reduced, because the feature model provides a objective reference point. Variation of performance among students is a complex task, because of confounding factors as motivation, previous experience, and problem solving skills.

Bean Shell exploratory programming
protoboard

types of software 
Type E

O termo feature é utilizado em diferentes tradições da Engenharia de Software. No contexto deste artigo, ele designa um recurso funcional das ferramentas Code Pad e JShell, em sentido semelhante ao uso em documentação técnica e em Sommerville (2015).
Reconhece-se, contudo, que o termo também possui acepções formais em Feature-Oriented Software Development (Kang et al., 1990; Batory et al., 2004) e em Behavior-Driven Development (North, 2006), onde descreve unidades de comportamento observável (feature files).

\section{Conclusion}

The Java Shell provides interactive and exploratory features that can improve the experience to programmers on all levels of maturity.

The use of professional tools on the classroom is a reality that calls for a more diverse set of tools, to accomodate a more diverse set of students.

All/many/some example of the benchmark can be reproduced using the Java Shell.

Code Pad has unique features, as the integration with visual models, and a simplified set of messages. 


Balance the power of the Java Shell with the easier adoption of BlueJ.

The pragmatic pedagogical lens adopted here emphasizes observable use: students explore, test, and learn through the specific feedback and interaction patterns that Code Pad or JShell provide.

Future work will formalize this comparison through feature modeling, allowing these pragmatic observations to be represented systematically

As a next step, we plan to formalize the comparative model using tools such as FeatureIDE, and to investigate other BlueJ features—especially visual modeling and object inspection—as subjects for future papers.



\bibliographystyle{plain}
\bibliography{refs}

\end{document}
