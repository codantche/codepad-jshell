\documentclass{article}
\usepackage{graphicx} 
\usepackage{plantuml}
\usepackage{url}

\usepackage{listings}
\usepackage{xcolor}

\lstdefinestyle{jshell}{
  language=Java,
  basicstyle=\ttfamily\small,
  backgroundcolor=\color{gray!5},
  frame=single,
  framerule=0.5pt,
  rulecolor=\color{gray!30},
  numbers=left,
  numberstyle=\tiny\color{gray},
  xleftmargin=1em,
  showstringspaces=false,
  tabsize=2,
  columns=flexible,
  moredelim=**[is][\color{blue}]{@}{@}, % for blue prompt like @jshell>
}

\lstset{style=jshell}

\title{Comparing Code Pad and JShell features}
\author{Marco Mangan\thanks{\texttt{marco.mangan@gmail.com}}}
\date{October 2025}

\begin{document}

\maketitle

\section{Introduction}

Learning to program computers is hard. 

Objects later

Objects first

Visual

BlueJ is \cite{barnes2008objects}\cite{bluej_paper}

Code Pad is

JShell is \cite{jshell_docs}

Comparing Code Pad and JShell features

Feature Modeling is

\section{Motivation}

This study is motivated by observations on the classroom of a undergraduate Computer Science course since the adoption of Java, BlueJ, and the Objects First approach.

Since the adoption of that approach, more than two decades have passed. The main factors observed on this period include: (a) an increase in computer literacy, (b) access to new devices and computer programs, and (c) the change of reference of what a computer program is.

BlueJ is a desktop application, with a look and feel that is contrasting with current applications. In particular, the multiple document interface causes the most distracting factor during classes.

Students usually prefer other programming environments, usually, Microsoft Visual Studio Code. Sometimes there is a pressure of senior students to move directly to a more professional editor. 

%Other editors include simple text editors, like Geany. No matter the editor, we can be sure that the Java Development Kit, that includes the Java Sheel, is available.

VS Code can be run locally or even on the cloud, through a web browser. This allows the students to use tablet devices and also offer a more familiar and professional looking user interface. Some students bring their own notebook or tablet devices, but most have access to more than one computer. The cloud-based VS Code will not need a separate Java and BlueJ installation and will keep all projects on the same cloud-based folder.

In this way, there is a need to offer a BlueJ-like experience for students that opt out of the BlueJ program. 

\section{Research Question}

The overall research question is broad: Could we have a BlueJ-like learning experience using professional tools?

BlueJ has many pedagogically founded features. Therefore we must compare the experience on each feature. This study focus on the interactive and exploratory learning experience offered by BlueJ Code Pad.

Therefore, the main question of this study is:
Could the use of the Java Shell provide students with a learning experience comparable to the one provided by the BlueJ Code Pad?

Even with this focus, we need to use sub-features to allow a direct comparision. By doing so, we provide a feature model that would help to understand the learning experience offered by both programs.

%, one of the many BlueJ pedagogically found features

%JShell was not available at the time that the BlueJ was proposed. Read-eval-loop shells were available since the adoption of interpreted languages, including the Unix bash shell, BASIC, and popular languages like Python and ECMAScript.


\section{Methodology}

Literature review


Benchmark with examples

Feature Oriented Domain Models

Comparing both programs


Most evidences are available on the paper repository \cite{mangan2025codepadjshell}.

\section{Student Expectations}

Get a job fast

Use professional tools

Write professional programs: mobile and Web clients, Web applications, games.

\section{Usage Scenarios}

Exercises from Barnes and Kolling

\subsection{Evaluate expressions}

BlueJ return value and type of each expression. 
The internal structure of the objects can be visually inspected in BlueJ. 

JShell returns only the value. Data types can be consulted using the command /vars.
JShell users must recur to introspection commands, which would be out of the scope of a first programming course. 
The CLI of Java Shell at least allow the use of ids to refer to expression values,  objects  and basic types.

\begin{lstlisting}
jshell> 4 + 5
$1 ==> 9
\end{lstlisting}

Method call
\begin{lstlisting}
jshell> "Wombat".substring(3, 5)
$2 ==> "ba"
\end{lstlisting}
Two-way integration with visual object model

After each compile

Importing packages

Variable declaration

Saving scripts

Conditional commands


\section{Feature Modeling}
Commonality and variability are

\section{Related Work}

At least on study points out the inadequacy of the Java Shell in the classroom
\cite{politz_minnes2018jshell}.

\section{Discussion}

JShell is a professional tool

BlueJ integration of many features.

BlueJ features not provided by Java Shell include class and object visual models and their integration with Code Pad.

In order to compare the potential experience, we compare features of the applications.

The need for a in class room is reduced, because the feature model provides a objective reference point. Variation of performance among students is a complex task, because of confounding factors as motivation, previous experience, and problem solving skills.


\section{Conclusion}

The Java Shell provides interactive and exploratory features that can improve the experience to programmers on all levels of maturity.

The use of professional tools on the classroom is a reality that calls for a more diverse set of tools, to accomodate a more diverse set of students.

All/many/some example of the benchmark can be reproduced using the Java Shell.

Code Pad has unique features, as the integration with visual models, and a simplified set of messages. 


Balance the power of the Java Sheel with the easier adoption of BlueJ.

Further studies are needed to complete the broader feature model. The evaluation of visual modeling extensions is next step on this research.


\bibliographystyle{plain}
\bibliography{refs}

\end{document}
